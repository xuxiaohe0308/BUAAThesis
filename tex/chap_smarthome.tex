\chapter{小型拟真智能家居实物实验平台}

\section{小型拟真智能家居实物实验平台需求调研}

\subsection{典型开源智能家居平台介绍}

\subsection{现有公开数据集调研}

近年来,智能家居越来越受到研究者的重视,学术界逐渐有一些研究者自行搭建智能家居实验环境,收集用于自己学术研究的数据集并将其公开,供后续研究使用。然而,这些学术研究的侧重点不同,导致其数据集通常不能覆盖所有智能家居的使用场景。本节调研了现有学术界面向智能家居场景的公开数据集,分析其数据特性及适用场景,为本文后续研究中的数据集选择,以及实验平台搭建需求分析做支撑。由于本文研究主要关注事件安全性问题,所用数据主要涵盖各智能设备状态流、控制中枢的事件日志、以及自动化规则等,故本节着重从上述方面进行调研。

英国牛津大学Birnbach等人在其研究\upcite{birnbach2019peeves}中收集了一组数据集并将其公开\upcite{peeves_dataset},他们在一个真实办公室环境内,使用路由器、智能手机、笔记本、树莓派以及各种高精度传感器电路模组构成了覆盖12种感知能力的传感器组合,并以不同的数量、不同的位置布置在办公室的各个位置。这些传感器不仅包括温度、湿度、光照强度、声音强度等对物理通道状态的感知,还包括接触传感器、开关等对设备状态的感知。在实验持续的14天内,办公室内的用户均手动地使用屋内各种设备,各个传感器不间断收集数据,并存储在本地U盘中。实验结束后研究者对数据进行了打包并公开。数据分传感器以csv文件的形式储存,由二元组$(t,v)$的形式组成,其中$t$为时间戳,$v$为传感器读数。此数据集的优势在于其完全利用自制传感器,传感器汇报率、精度较高,便于进行细粒度的数据分析以及复杂机器学习模型的训练,并且其采样持续时间长,传感器类型丰富;然而,其并没有引入智能家居平台作为中枢,即其不能体现智能家居的触发-响应特性,故不适用于事件依赖的相关研究。

类似地,Chimamiwa等人\upcite{chimamiwa2021multi}公开了其智能家居传感器序列数据集,该数据集从多个环境传感器收集而来,主要目的为捕获人类的日常生活活动。传感器类型包括温湿度、开关、光传感器等,收集持续时间为6个月,采样率为1Hz。同时,此数据集包括一个公寓内多个房间采集的数据。数据格式方面,传感器以数据库的形式分别存储在五个表中,并以csv文件的形式公开。传感器数据按整型数据和浮点型数据分别存储在两个表中,剩余三个表则为传感器名称等基本信息。每一数据条目由四元组$(id,t,v,s)$的形式组成,其中$t$为时间戳,$v$为传感器读数,$s$为传感器名称。从使用场景来看,此数据集相比上述牛津大学数据集更加丰富,且时间跨度更大。但此数据集每个房间内的传感器密度、种类较小,且同样没有考虑智能家居的触发-响应特性。

此外,华盛顿州立大学自适应系统高级研究中心(Center for Advanced Studies in Adaptive Systems,CASAS)公开了其智能家居环境内采集的传感器数据集\upcite{casas},通过使用数据分析及人工智能技术来对家居环境进行解释、建模和预测,主要用于用户活动感知、自动健康诊断、节能家居自动化等领域,目标为改善用户的生活。此数据集主要面向用户活动引起的智能家居设备状态改变活动,采集的数据为总日志条目的形式,每一条目以三元组$(t,d,s)$组成,其中中$t$为时间戳,$d$为设备名称,$s$为设备状态。此数据集相比上述牛津大学数据集,用户活动以及场景更加丰富,但传感器精度、采样率、种类略有不足,故其更适合用于更上层的建模分析工作,如用户活动感知等。同样地,此数据集亦没有引入智能家居平台作为中枢,没有体现触发-响应特性。

除此之外,一些民间技术爱好者将数据集分享至知名数据分析竞赛平台Kaggle\upcite{kaggle}上,这些数据集主要关注智能家居场景中某单一方面的数据,如各设备用电功率数据、智能家居控制指令数据等。总体来说,这些数据集涵盖事件类型、物理通道种类较少,并且缺少数据采集时的实验细节,不适合用于学术研究。

\subsection{实验平台需求分析}

上述介绍的现有智能家居公开数据集,绝大多数局限于传感器读数的记录,即在一个设定好的智能家居场景内安装传感器,并按不同的时间间隔连续采集数据。在事件指纹的研究方面,Peeves\upcite{birnbach2019peeves}进行的是类似的工作,故本文在事件指纹方面的研究可用此数据集进行;而由于本文还需要对事件依赖安全进行研究,在显式事件依赖方面,需要考虑智能家居场景中设定的触发-响应形式的自动化规则吗,而现有数据集均未考虑这一设定;而在隐式事件依赖方面,需要细粒度、定量地建模不同事件对物理通道的影响,故需要触发单一事件并连续收集物理通道的影响,而现有数据集中多个事件可能同时或以很小的时间间隔触发,造成其物理通道影响杂糅交错,会造成后续数据分析工作的误差。因此,现有公开数据集均不适合作为本文事件依赖相关研究的支撑。

此外,除上述公开数据集的研究之外,现有智能家居安全相关研究(尤其是事件指纹、事件依赖等相关研究),绝大多数都未采用现成公开的数据集,而是自行搭建智能家居环境并设计应用场景来进行数据采集,导致这些数据集的泛用性较低,使用场景局限。同时,这些研究绝大多数均未公开自己的数据,这使得后续研究仍需重新设计并采集新的数据集,导致工作量加大,同时也无法进行有效、公平的横向对比。

综上所述,为了使智能家居场景灵活化,数据收集的过程方便化,本文设计并实现了小型拟真智能家居实物实验平台。本实验平台的目标为支撑本文后续研究中需要的数据收集工作,并作为后续多样的相关研究工作提供灵活的数据来源。具体来说,本实验平台的设计需求如下:

\textbf{(1)拟真性:}首先,不同于现有数据集并未引入真实的智能家居平台,本实验平台需要尽量地贴合真实的智能家居使用场景。具体来说分为三个方面:在控制中枢方面,本实验平台需要引入一个真实的智能家居中枢系统,并设计合理的自动化规则来模拟用户的使用场景;同时,在设备方面,本实验平台需要设置贴合真实场景的传感器与智能设备组合;此外,本实验平台还需模拟真实智能家居环境的其他特性,如家居物理环境、数据传输方式、协议类型等。

\textbf{(2)灵活性:}其次,本实验平台需要具有灵活多变的场景模拟能力。具体来说,本实验平台需要能够灵活改变各类用户属性及配置,如自动化规则等,用于满足用户的各种使用场景,以满足后续不同研究工作的数据需求。同时,本实验平台需要具有一定的设备扩展能力,即在后续出现新的需求时,可在不影响现有系统的前提下添加新的设备,来扩展平台的使用场景。

\textbf{(3)易用性:}最后,本实验平台需要具有丰富的上层功能设计,以使数据采集的过程更加方便、直观。具体来说,本实验平台需要为数据采集过程设计并实现丰富的前台功能,研究者只需输入与实验设置有关的一些基本信息(如设定哪些规则、触发哪些事件、收集哪些时间段的数据等),实验平台即可自动化完成数据采集与打包工作,无需用户向智能家居平台或数据库等后端应用进行对接。同时,本实验平台需要设计直观、易操作的前端用户UI界面,实时展示状态的同时也能方便用户控制。

\section{实验平台设计}

基于上述实验平台需求,本节从架构、功能两方面分别介绍本实验平台的设计。

\subsection{硬件架构设计}

架构图

智能家居平台:树莓派+Home assistant,mysql数据库

设备:DIY设备/传感器模组,ESP32模块,I2C或GPIO连接

\subsection{软件架构设计}

web app

1. 连接配置模块
2. Home Assistant RESTful API对接模块
3. 数据库自动化读取模块

\subsection{实验平台功能设计}

1. 自定义服务发送时间和状态
2. 自定义数据收集时间段
3. 智能家居状态实时展示

\section{实验平台实现}

\subsection{感知层设备选择与部署}

\subsection{智能家居平台部署}

\subsection{自动化数据收集功能实现}

\section{平台使用方法与案例}
