\chapter{小型拟真智能家居实物实验平台}

\section{小型拟真智能家居实物实验平台需求调研}

%\subsection{典型开源智能家居平台介绍}

\subsection{现有公开数据集调研}

近年来,智能家居越来越受到研究者的重视,学术界逐渐有一些研究者自行搭建智能家居实验环境,收集用于自己学术研究的数据集并将其公开,供后续研究使用。然而,这些学术研究的侧重点不同,导致其数据集通常不能覆盖所有智能家居的使用场景。本节调研了现有学术界面向智能家居场景的公开数据集,分析其数据特性及适用场景,为本文后续研究中的数据集选择,以及实验平台搭建需求分析做支撑。由于本文研究主要关注事件安全性问题,所用数据主要涵盖各智能设备状态流、控制中枢的事件日志、以及自动化规则等,故本节着重从上述方面进行调研。

英国牛津大学Birnbach等人在其研究\upcite{birnbach_peeves_2019}中收集了一组数据集并将其公开\upcite{peeves_dataset},他们在一个真实办公室环境内,使用路由器、智能手机、笔记本、树莓派以及各种高精度传感器电路模组构成了覆盖12种感知能力的传感器组合,并以不同的数量、不同的位置布置在办公室的各个位置。这些传感器不仅包括温度、湿度、光照强度、声音强度等对物理通道状态的感知,还包括接触传感器、开关等对设备状态的感知。在实验持续的14天内,办公室内的用户均手动地使用屋内各种设备,各个传感器不间断收集数据,并存储在本地U盘中。实验结束后研究者对数据进行了打包并公开。数据分传感器以csv文件的形式储存,由二元组$(t,v)$的形式组成,其中$t$为时间戳,$v$为传感器读数。此数据集的优势在于其完全利用自制传感器,传感器汇报率、精度较高,便于进行细粒度的数据分析以及复杂机器学习模型的训练,并且其采样持续时间长,传感器类型丰富;然而,其并没有引入智能家居平台作为中枢,即其不能体现智能家居的触发-响应特性,故不适用于事件依赖的相关研究。

类似地,Chimamiwa等人\upcite{chimamiwa2021multi}公开了其智能家居传感器序列数据集,该数据集从多个环境传感器收集而来,主要目的为捕获人类的日常生活活动。传感器类型包括温湿度、开关、光传感器等,收集持续时间为6个月,采样率为1Hz。同时,此数据集包括一个公寓内多个房间采集的数据。数据格式方面,传感器以数据库的形式分别存储在五个表中,并以csv文件的形式公开。传感器数据按整型数据和浮点型数据分别存储在两个表中,剩余三个表则为传感器名称等基本信息。每一数据条目由四元组$(id,t,v,s)$的形式组成,其中$t$为时间戳,$v$为传感器读数,$s$为传感器名称。从使用场景来看,此数据集相比上述牛津大学数据集更加丰富,且时间跨度更大。但此数据集每个房间内的传感器密度、种类较小,且同样没有考虑智能家居的触发-响应特性。

此外,华盛顿州立大学自适应系统高级研究中心(Center for Advanced Studies in Adaptive Systems,CASAS)公开了其智能家居环境内采集的传感器数据集\upcite{casas},通过使用数据分析及人工智能技术来对家居环境进行解释、建模和预测,主要用于用户活动感知、自动健康诊断、节能家居自动化等领域,目标为改善用户的生活。此数据集主要面向用户活动引起的智能家居设备状态改变活动,采集的数据为总日志条目的形式,每一条目以三元组$(t,d,s)$组成,其中中$t$为时间戳,$d$为设备名称,$s$为设备状态。此数据集相比上述牛津大学数据集,用户活动以及场景更加丰富,但传感器精度、采样率、种类略有不足,故其更适合用于更上层的建模分析工作,如用户活动感知等。同样地,此数据集亦没有引入智能家居平台作为中枢,没有体现触发-响应特性。

除此之外,一些民间技术爱好者将数据集分享至知名数据分析竞赛平台Kaggle\upcite{kaggle}上,这些数据集主要关注智能家居场景中某单一方面的数据,如各设备用电功率数据、智能家居控制指令数据等。总体来说,这些数据集涵盖事件类型、物理通道种类较少,并且缺少数据采集时的实验细节,不适合用于学术研究。

\subsection{实验平台需求分析}

上述介绍的现有智能家居公开数据集,绝大多数局限于传感器读数的记录,即在一个设定好的智能家居场景内安装传感器,并按不同的时间间隔连续采集数据。在事件指纹的研究方面,Peeves\upcite{birnbach_peeves_2019}进行的是类似的工作,故本文在事件指纹方面的研究可用此数据集进行;而由于本文还需要对事件依赖安全进行研究,在显式事件依赖方面,需要考虑智能家居场景中设定的触发-响应形式的自动化规则吗,而现有数据集均未考虑这一设定;而在隐式事件依赖方面,需要细粒度、定量地建模不同事件对物理通道的影响,故需要触发单一事件并连续收集物理通道的影响,而现有数据集中多个事件可能同时或以很小的时间间隔触发,造成其物理通道影响杂糅交错,会造成后续数据分析工作的误差。因此,现有公开数据集均不适合作为本文事件依赖相关研究的支撑。

此外,除上述公开数据集的研究之外,现有智能家居安全相关研究(尤其是事件指纹、事件依赖等相关研究),绝大多数都未采用现成公开的数据集,而是自行搭建智能家居环境并设计应用场景来进行数据采集,导致这些数据集的泛用性较低,使用场景局限。同时,这些研究绝大多数均未公开自己的数据,这使得后续研究仍需重新设计并采集新的数据集,导致工作量加大,同时也无法进行有效、公平的横向对比。

综上所述,为了使智能家居场景灵活化,数据收集的过程方便化,本文设计并实现了小型拟真智能家居实物实验平台。本实验平台的目标为支撑本文后续研究中需要的数据收集工作,并作为后续多样的相关研究工作提供灵活的数据来源。具体来说,本实验平台的设计需求如下:

\textbf{(1)拟真性:}首先,不同于现有数据集并未引入真实的智能家居平台,本实验平台需要尽量地贴合真实的智能家居使用场景。具体来说分为三个方面:在控制中枢方面,本实验平台需要引入一个真实的智能家居中枢系统,并设计合理的自动化规则来模拟用户的使用场景;同时,在设备方面,本实验平台需要设置贴合真实场景的传感器与智能设备组合;此外,本实验平台还需模拟真实智能家居环境的其他特性,如家居物理环境、数据传输方式、协议类型等。

\textbf{(2)灵活性:}其次,本实验平台需要具有灵活多变的场景模拟能力。具体来说,本实验平台需要能够灵活改变各类用户属性及配置,如自动化规则等,用于满足用户的各种使用场景,以满足后续不同研究工作的数据需求。同时,本实验平台需要具有一定的设备扩展能力,即在后续出现新的需求时,可在不影响现有系统的前提下添加新的设备,来扩展平台的使用场景。

\textbf{(3)易用性:}最后,本实验平台需要具有丰富的上层功能设计,以使数据采集的过程更加方便、直观。具体来说,本实验平台需要为数据采集过程设计并实现丰富的前台功能,研究者只需输入与实验设置有关的一些基本信息(如设定哪些规则、触发哪些事件、收集哪些时间段的数据等),实验平台即可自动化完成数据采集与打包工作,无需用户向智能家居平台或数据库等后端应用进行对接。同时,本实验平台需要设计直观、易操作的前端用户UI界面,实时展示状态的同时也能方便用户控制。

\section{实验平台设计}

基于上述实验平台需求,本节从硬件架构、软件架构两方面分别介绍本实验平台的设计。本平台的整体架构如图\ref{fig:sh_intro}所示。

\begin{figure}[!h]
	\centering
	\includegraphics[width=.95\textwidth]{pic/chap3/sh\_intro}
	\caption{小型拟真智能家居实物实验平台架构图}
	\label{fig:sh_intro}
\end{figure}

\subsection{硬件架构设计}

为满足设计需求中的拟真性,本实验平台引入了真实的智能家居平台作为中枢环境,用于处理与智能家居相关的逻辑。同时,为满足设计需求中的灵活性,即设备的多样性和可扩展性,实验平台选取了DIY平台Home assistant\upcite{homeassistant}。该平台为开源平台,通过制定一套原生API逻辑与各种不同的智能家居设备通信。目前已有许多商用智能家居品牌的设备支持通过此API连接到Home assistant,如小米、三星等。

Home assistant提供了三种部署形式:Python模组、容器以及操作系统。考虑到本实验平台除Home assistant外还需同步运行一些软件逻辑,平台使用一台树莓派4B作为家居中枢,安装Linux系统并以Python模组的方式部署智能家居平台。由于树莓派具有小型化、轻量化的优点,其较为适合本文的小型拟真智能家居平台,且树莓派4B的计算性能对于Home assistant平台已经足够。为部署Home assistant平台,还需配置sql数据库作为智能家居日志的存储器。考虑到树莓派的存储性能有限,本平台配置了一台远程mysql数据库服务器主机,使用其远程存储智能家居日志,便于后续数据收集的过程。

除智能家居中枢外,本平台还包括一系列小型智能家居设备与传感器。为满足小型化、灵活性的需求,平台采用了DIY设备与传感器模块的形式,这是由于这些模块足够轻量级。设备通信方面,需要将这些设备与传感器模块接入Wi-Fi并与树莓派通信,故平台采用了ESP32 Wi-Fi模组的方式。此模块是一个拥有数个GPIO接口的可编程Wi-Fi开发模块。进一步地,为使设备与传感器能通过Home assistant原生API与其联通,本平台在ESP32上烧录开源ESPHome\upcite{esphome}软件,该软件可以将GPIO接口采集到的数据转化为API支持的形式,自动通过Wi-Fi与Home assistant联通。

传感器方面,相比于传统商业化智能家居传感器,本平台采用的传感器模组均精度较高,且可达到较高的采样率。通常来说,这些高性能传感器使用$^2$总线的形式进行板间通信。I$^2$C是一种使用一根时钟线和一根数据线即可实现的数字通信总线协议,且可支持多传感器并联。ESPHome开源程序原生在原有GPIO基础上实现I$^2$C,只需在配置文件中定义$^2$引脚号即可。综上所述,使用ESP32并烧录ESPHome的方式能极为方便地使各种类型的设备和传感器模组与Home assistant联通,可以保证设备的多样性和扩展性,故进一步支持了设计需求中的灵活性。

\subsection{软件架构设计}

在上述硬件的基础上,本实验平台还需软件支撑,用于满足设计需求中的易用性。具体来说,为满足实验平台数据收集的功能,支撑后续研究工作,本实验平台的软件部分计划实现下述基本功能:

\begin{enumerate}
	\item 自定义事件发起时间和种类:为模拟智能家居的各种使用场景,需要自定义地发起事件或事件序列,并能够预先设定事件发起规则;
	\item 自定义数据收集时间段:为满足数据收集需求,需要自定义数据收集的时间段,同步完成预处理,输出能够直接进行数据分析的文件格式(如csv等)。
\end{enumerate}

此外,考虑到软件的可移植与便携性,本平台选择Web应用程序的形式,将后端运行在树莓派上,使树莓派同时运行Home assistant与Web应用程序,如此以来即可将客户端的要求降到最低。

本平台的软件部分主要包含以下四个模块:(1)用户前端模块;(2)Home assistant通信模块;(3)数据库通信模块;(4)核心逻辑模块。

用户前端模块用于为用户提供连接配置接口,其中用户需要输入Home assistant与Mysql数据库的主机、端口等信息,Web应用程序接收到上述连接信息后尝试连接,测试系统的联通性。同时,由于Home assistant具有用户认证功能,正常用户需要输入用户名与口令进入其前端页面,而本Web应用程序通过Home assistant提供的原生RESTful API通信,其认证过程中通过请求头的token字段实现,故此用户前端模块还包括读取用户提供的token文件的功能。

Home assistant通信模块通过原生RESTful API\upcite{restfulapi}与Home assistant进行交互。通过此API可以以GET或POST请求的方式完成大部分智能家居核心功能,如读取设备状态、发起事件(服务)来控制设备等。

数据库通信模块用于远程连接Mysql数据库,并使用SQL参数化查询的方式读取数据库中的智能家居日志记录。

核心逻辑模块在上述三个基础模块的基础上搭建,用于实现Web应用程序的上层功能。其中,为实现自定义事件发起时间和种类,Web应用程序首先通过调用API获取Home assistant的所有已知服务和设备,同时接收用户设定,以定时后台任务的方式来控制Home assistant;为实现自定义数据收集时间段的功能,Web应用程序接收用户设定,并将时间等参数传给数据库通信模块。

\section{实验平台实现}

本节介绍本实验平台在实现上的技术细节。具体来说分为两方面,首先介绍实验平台在智能家居设备上的选择、供电方式以及位置安排等,之后介绍实验平台软件方面的代码实现细节和用户接口定义。

\subsection{设备选择与部署}

\subsection{软件功能实现}


\section{平台使用方法与案例}
