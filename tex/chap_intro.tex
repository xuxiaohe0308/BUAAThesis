% !TeX root = ../Template.tex
% [绪论]
% 此处为本LaTeX模板的简介
\chapter{绪论}

\section{课题来源}

国家自然科学基金面上项目:基于多源事件复合推演的物联网安全溯源与异常检测机理研究。

\section{研究背景}

智能家居(Smart Home)(又称为自动化家居、智慧家居)这一概念在1984年由American Association of House Builders提出[1],是一种物联网(Internet of Things,IoT)的典型应用场景,由一系列控制家居环境的智能设备组成。起初智能家居被用于照明和供暖、制冷等系统的自动控制,而随着技术发展,目前智能家居基本涉及到了用户房屋内几乎所有种类的部件(包括门窗、电灯、智能音箱、各种开关、各种传感器等等)。此外,智能家居可以实现对房屋环境的实时监控,以及对智能设备的远程控制,用户以直接独立操作设备或设定自动化规则的形式参与其中[2]。通常情况下,智能家居提供的功能包括舒适度、安全性、可靠性、远程控制能力和能源节约能力[3]。

从结构上来讲,智能家居可用以下三层结构来描述[4]:感知层、网络层、应用层。感知层包括智能设备和传感器,用于感知物理环境的状态;网络层包括网关、移动设备和服务器等硬件设施,通过构建家庭网络来为设备间通信提供可能;应用层包括在移动设备或服务器上运行的App等形式的智能逻辑,负责提供用户UI接口,或通过自动化规则进行智能决策。
作为智能家居系统运转的基础,感知能力保证了家居环境状态能够被各类传感器和设备实时收集,从而进一步为用户和应用层逻辑(通常以App的形式)提供准确信息以供决策。现有大多数成熟的智能家居平台(如SmartThings[6]与Home Assistant[7])为事件驱动的(Event-driven),即通过事件实现信息传递,其中事件(Event)是由设备发出的一种网络消息,用于描述相关的设备状态或环境信息[2]。

智能家居系统通过事件总线(Event Bus)[7]来收集汇总事件,通过发布(Publish)-订阅(Subscribe)机制来控制事件的流向,设备通过发布操作向事件总线添加新事件,应用层逻辑通过订阅操作在事件总线中登记,当有相应事件的发布时,事件总线将这一消息转发给所有订阅过此事件的应用层逻辑[2]。图1为一个简单的智能家居事件示意图,当智能设备“灯”被用户或被系统远程打开时,其会向智能家居云发布一个“开灯”事件,云中运行的App通过订阅此事件来感知智能电灯的状态,从而进一步做出智能决策。总体来说,事件是现实生活中发生事件的抽象表示,并将物理空间中的智能设备实体映射到网络空间中,使设备能够参与智能决策的过程、提供信息或接受指令[8],是智能家居系统的核心要素。

事件是智能家居实现感知能力的基础,然而事件存在诸多安全隐患,这些安全隐患可能导致系统出现逻辑上的异常,从而出现意外行为;或是被攻击者利用,使敏感设备受控。文献[9]整理总结了智能家居事件可能出现的安全问题,其中包括事件丢失、事件截断、事件错误、虚假事件等。文献[10]以SmartThings为例指出,事件消息是缺乏保护的,一旦攻击者得到设备ID等敏感信息便可轻易地伪造事件消息,此攻击称为事件欺骗攻击(Event Spoofing Attack)。此研究进一步指出,绝大多数设备均会向系统发布事件,这些事件均会面临被伪造的风险。事件欺骗攻击的实施手段多种多样,包括节点妥协[15]、恶意App植入[16]、中间人攻击[17]、流量包伪造[18]以及超声波语音助手攻击[19][20][21]等攻击方式,故对攻击者而言非常容易实施。

上述伪造的虚假事件会造成系统对设备状态的错误感知,导致系统执行错误的决策,而通过伪造事件来有目的性地触发自动化规则,攻击者可以在未经授权的前提下控制一些敏感设备。例如,攻击者可以利用“用户到家则解锁门锁”的自动化规则,向系统注入“到家”的虚假事件,从而使门锁打开。以上案例说明,诸如事件欺骗攻击一类的攻击可以使系统产生虚假的事件,从而使系统自动化规则被意外触发,使敏感设备受控。

除此之外,由于智能家居的使用环境以及用户或App设置的关联规则的影响,事件之间存在错综复杂的因果性触发规律,构成了极为复杂的事件依赖关系。其中跨应用程序干扰[11][12]由不同App对同一设备进行同时操作引起,会造成命令冲突,引起意外的系统行为,使命令未按照用户期望执行;而危险事件依赖[13][14]可能来源于某些隐式事件依赖,进而引起危险的触发动作,造成敏感设备受控,或家居环境的危险物理状态,如扫地机器人触发动作传感器,进而使系统认为有人在家,进而使门锁打开。上述两种安全隐患的共同之处在于,其均违背了智能家居的正常运行模式,故本文将上述两种安全隐患统称为“异常事件依赖”。

综上所述,智能家居事件的安全问题会造成严重的安全后果,故针对智能家居事件安全的研究是至关重要的。本文拟从上述讨论触发,从虚假事件、异常事件依赖两方面安全问题出发,展开相关虚假事件检测、事件依赖提取、异常事件依赖检测方面的研究,旨在提高智能家居事件的安全性,保证智能家居的正确感知能力与逻辑正确性,从而为整体系统的安全运行提供基础保障。

\section{研究意义}

针对上述讨论,本论文的研究意义主要表现在以下三方面:

\begin{enumerate}
	\item 为了解决虚假事件问题,有效检测事件欺骗攻击,本论文研究一种基于物理上下文的事件真实性验证方案,对现有方案的准确率进行改进,同时考虑到实际部署情况,通过缩减传感器数量达到轻量化的目的;
	\item 为了对事件依赖进行全面提取和展示,本论文研究一种基于复合上下文建模的事件依赖图构建方案,实现了事件依赖的全面、准确提取和直观表示,提升用户对系统的感知能力;
	\item 为了检测异常事件依赖,本论文研究一种基于事件依赖特征的异常检测方法,保证了系统中的事件依赖关系在语义上是准确的、安全的。
\end{enumerate}
%%----------------------

\section{本文研究内容以及论文构成}

本文基于智能家居的上下文信息,研究事件安全性相关问题,旨在提升智能家居系统的安全性。具体来说,考虑可能出现的虚假事件,设计并实现一种基于物理上下文的事件真实性验证方法,保证智能家居事件的真实性;考虑可能存在的异常事件依赖,设计并实现基于复合上下文建模的事件依赖图构建方法,以及基于事件依赖特征的异常检测方法,实现事件依赖的全面提取与静态异常检测,分析系统可能出现的安全隐患,提高智能家居安全性。

本论文的研究内容如下:

\begin{enumerate}
	\item 智能家居自动化数据收集实物平台搭建:
	
	\item 基于物理上下文的事件真实性验证:研究通过智能家居事件物理上下文建立事件指纹的方法,并利用指纹实现事件验证,优化现有方案效果;
	\item 基于复合上下文建模的事件依赖图构建:研究通过逻辑上下文提取显式事件依赖的方法,研究通过事件物理上下文的事件物理关系建模方法,利用物理关系提取隐式事件依赖,并实现事件依赖图的构建;
	\item 基于事件依赖特征的异常检测:研究事件依赖图结构特征的提取方法,研究事件依赖语义的表达方法以及语义特征提取方法,将事件依赖向量化,研究用于事件依赖异常检测的机器学习模型,最终实现基于事件依赖特征的异常检测。
\end{enumerate}

